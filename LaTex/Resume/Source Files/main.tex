% Harsh Dani - Resume
% 7/26/2023
% Reference:
% Debarghya Das (http://debarghyadas.com)
\documentclass[]{resume}
\begin{document}
%     TITLE NAME
\namesection{Harsh Dani}
{\urlstyle{same}
	\faEnvelope \href{mailto:harshdani@myyahoo.com}{harshdani@myyahoo.com}\\
	\faGithub \href{https://github.com/harshtechnoboy}{github.com/harshtechnoboy}\\
	\faLinkedinSquare \href{https://www.linkedin.com/in/harshtechnoboy}{linkedin.com/in/harshtechnoboy}
}
%     COLUMN ONE
\begin{minipage}[t]{0.34\textwidth} 
%     SKILLS
\section{SKILLS} 
\subsection{Programming Languages:}
C++, Java, Python, Visual Basic \\
\subsection{Frontend Development:}
CSS3, HTML5 \\
\subsection{AI / Machine Learning:}
Pandas, Plotly \\
\subsection{Database Management:}
MySQL \\
\subsection{Software:}
Autodesk 3DS Max \& AutoCAD, \\Adobe After Effects, Illustrator,\\ Lightroom, Photoshop \& Premiere Pro, Arduino, Blender, Canva, Figma, Git, Linux  \\
%     LANGUAGES
\section{Languages}
\subsection{English:}
IELTS Band 7.0 Overall \\
\subsection{German:}
Certified B2 Level \\
Studied till C1.1 Level
\sectionsep
%     COURSEWORK
\section{Coursework}
\subsection{Data Structures \& Algorithms}
\vspace{1em}
\subsection{Databases \& Big Data} 
\vspace{1em}
\subsection{Python Programming} 
\vspace{1em}
\subsection{Mathematical Foundations} 
\vspace{1em}
\subsection{Computer Science Lab} 
\sectionsep
%     COLUMN TWO
\end{minipage} 
\hfill
\begin{minipage}[t]{0.65\textwidth} 
%     EDUCATION
\section{Education}
\workplace{GISMA University of Applied Sciences}{September 2022 – present}\\
\position{BEng Software Engineering}{Potsdam, Germany}
\sectionsep
\workplace{HVB Global Academy}{May 2019 – April 2022} \\
\position{Cambridge International A Level}{Mumbai, India}
\vspace{\topsep} 
\begin{tightemize}
\item Chemistry 
\item Computer Science
\item English
\item Mathematics
\end{tightemize}
\sectionsep
\workplace{HVB Academy}{March 2018 – May 2019} \\
\position{Indian Certificate of Secondary Education}{Mumbai, India}
\begin{tightemize}
\item Chemistry
\item Computer Application
\item English
\item Mathematics
\item Physics
\end{tightemize}
\sectionsep
%     PROJECTS
\section{Projects}
\runsubsection{Tic Tac Toe Game:}
\descript{}
Tic-Tac-Toe game using the Tkinter library in Python. It has a board represented by a 3x3 grid of boxes. Each player take turns by clicking on the empty boxes to mark their moves. It checks for a winner after each turn, the first player to get three in a row wins the game. In case the board is full, the game ends in a tie. It can even be reset.
\sectionsep 

\runsubsection{Calculator Application:}
\descript{}
Basic calculator application using the Tkinter library in Python. It creates a GUI window with buttons for digits, operators, and special functions. It supports keyboard input as well. The user can enter expressions, perform arithmetic operations and get the results. It provides a simple yet functional calculator interface.
\sectionsep

\runsubsection{Rock Paper Scissors Game:}
\descript{}
A basic implementation of the game "Rock, Paper, Scissors" played between a user and a computer. The user selects their choice, and the computer generates a random choice. The code determines the winner based on the rules: rock beats scissors, paper beats rock, and scissors beat paper. The scores of the user and computer are recorded, and the game continues until the user decides to quit. Finally, the code displays the total number of wins for the user and the computer.
\sectionsep

\end{minipage} 

\end{document}  
